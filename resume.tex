\documentclass{resume} % Use the custom resume.cls style

\usepackage[left=0.4 in,top=0.4in,right=0.4 in,bottom=0.4in]{geometry} % Document margins
\newcommand{\tab}[1]{\hspace{.2667\textwidth}\rlap{#1}} 
\newcommand{\itab}[1]{\hspace{0em}\rlap{#1}}

%----------------------------------------------------------------------------------------
%	PERSONAL INFO
%----------------------------------------------------------------------------------------

\name{Nafis Azizi Riza}
\address{\href{mailto:nafisredzone@gmail.com}{nafisredzone@gmail.com} \\ \href{tel:+6281336825409}{+6281336825409} \\ \href{https://nafisazizi.com}{nafisazizi.com} \\ \href{https://github.com/nafisazizir}{github.com/nafisazizir} \\ \href{https://linkedin.com/in/nafisazizi/}{linkedin.com/in/nafisazizi/}}

\begin{document}

%----------------------------------------------------------------------------------------
%	EDUCATION SECTION
%----------------------------------------------------------------------------------------

\vspace{-0.5em}
\begin{rSection}{Education}

{\bf Universitas Indonesia} \hfill {Aug 2021 - Jun 2025}
\vspace{-0.75em}
\begin{itemize}
\itemsep -7pt {}
\item Bachelor of Computer Science / Sarjana Ilmu Komputer (S.Kom), Cumulative GPA: 3.75/4.0
\item Relevant Coursework: Data Structure \& Algorithm, Platform Based Development, Database, Operating System, Computer Organization
 \end{itemize}
 \vspace{-0.5em}

{\bf National University of Singapore} \hfill {Jan 2023 - May 2023}
\vspace{-0.75em}
\begin{itemize}
\itemsep -7pt {}
\item Exchange students under ASEAN University Network Program
\item Coursework: Design and Analysis of Algorithm, Interaction Design, Computer Network, Software Engineering \& OOP, Programming Methodology II
 \end{itemize}
 \vspace{-0.5em}


\end{rSection}

%----------------------------------------------------------------------------------------
% SKILLS SECTION
%----------------------------------------------------------------------------------------
\begin{rSection}{SKILLS}

\begin{tabular}{ @{} >{\bfseries}l @{\hspace{6ex}} l }

Programming Languages & C++, Python, Java, Kotlin \\
Web/Mobile Development & React, Typescript, Javascript, HTML, CSS, Django, Flutter, Dart \\
Data, AI/ML & MySQL, PostgreSQL, MongoDB, TensorFlow, Keras, PyTorch, Scikit-Learn \\

\end{tabular}\\
\vspace{-0.75em}
\end{rSection}

%----------------------------------------------------------------------------------------
% EXPERIENCES SECTION
%----------------------------------------------------------------------------------------
\begin{rSection}{EXPERIENCE}

\textbf{Teaching Assistant} \hfill Aug 2022 - Jan 2023 \\
Universitas Indonesia \hfill \textit{Depok, Indonesia}
\vspace{-0.75em}
\begin{itemize}
\itemsep -7pt {}
\item Instructed and supported a group of 8-9 students in an introduction to programming in Python, resulting in improved student performance and engagement.
\item Created and assessed labs and programming assignments, tracking measurable progress and observing an average grade improvement of 75\% among students.
\item Facilitated weekly lab meetings with an average attendance of approximately 30 students, fostering a collaborative and interactive learning environment through hands-on exercises.
 \end{itemize}
 \vspace{-0.5em}

\textbf{Project Dev Intern, Business Dev Dept.} \hfill Oct 2022 - Dec 2022 \\
AIESEC in Universitas Indonesia \hfill \textit{Depok, Indonesia}
\vspace{-0.75em}
\begin{itemize}
\itemsep -7pt {}
\item Analyzed the AIESEC in UI program, Lead Series, and provided recommendations for improvement.
\item Led the planning process for a new initiative, the scholarship program, including creating the deck and proposal.
 \end{itemize}
 \vspace{-0.5em}

\textbf{Data Science Academy Staff} \hfill Apr 2022 - Nov 2022 \\
Data Science Academy - COMPFEST 14 \hfill \textit{Depok, Indonesia}
\vspace{-0.75em}
\begin{itemize}
\itemsep -7pt {}
\item Developed the pre-module for the Data Science Academy and prepared ToR for the speakers.
\item Assessed and provided constructive feedback on the curriculum to ensure its effectiveness and relevance, leading to improvements in content delivery.
 \end{itemize}
 \vspace{-0.5em}


\end{rSection} 

%----------------------------------------------------------------------------------------
%	PROJECTS SECTION
%----------------------------------------------------------------------------------------

\begin{rSection}{PROJECTS}

{\bf MatkulGue, University Course Catalog, Timetable Builder, and Course Planner}
\vspace{-0.75em}
\begin{itemize}
\itemsep -7pt {}
\item Developed a MERN stack web application for creating a course catalog and building personalized timetables
\item Scraped the course information from the university official websites and setting up the database
\item Improved scheduling efficiency and received positive user feedback on the user-friendly interface and unique feature
 \end{itemize}
 \vspace{-0.5em}

{\bf ACB-ISBE, Conference Proceeding App, Web \& Mobile App Version}
\vspace{-0.75em}
\begin{itemize}
\itemsep -7pt {}
\item Designed and developed an app to enable conference participants to access various informations of the conference using Python Django and Flutter.
\item Handled the design and processing of the app's database, and implemented a robust search feature
 \end{itemize}
 \vspace{-0.5em}

{\bf SIREST, Fullstack Web App for Restaurant Order and Delivery}
\vspace{-0.75em}
\begin{itemize}
\itemsep -7pt {}
\item Designed and implemented the database schema using PostgreSQL.
\item Built and deployed the websites by utilizing Python Django, ensuring seamless user experience and functionality.
 \end{itemize}
 \vspace{-0.5em}

{\bf Prediction Starcraft II Matches, Generate ML Model to Predict Matchup of Each Games}
\vspace{-0.75em}
\begin{itemize}
\itemsep -7pt {}
\item Conducted EDA and built a predictive machine learning model to forecast matchups in Starcraft II matches.
 \end{itemize}
 \vspace{-0.5em}

{\bf Climate Change Analysis, Data Analysis (EDA) Dashboard About Climate Change in Indonesia}
\vspace{-0.75em}
\begin{itemize}
\itemsep -7pt {}
\item Conducted EDA on a climate change dataset in Indonesia, by utilizing Python libraries, resulting in the identification of key trends and patterns in climate change data.
 \end{itemize}
 \vspace{-0.5em}


\end{rSection} 

%----------------------------------------------------------------------------------------
% ACHIEVEMENTS SECTION
%----------------------------------------------------------------------------------------
\begin{rSection}{Achievements}

\begin{itemize}
\itemsep -7pt {}
\item Finalist of Information technology Business Competition, Gemastik 2022
\item UI RISE Scholarship Awardee
 \end{itemize}
 \vspace{-0.5em}


\end{rSection}


\end{document}
